\documentclass[times, utf8, zavrsni]{fer}
\usepackage{booktabs}

\begin{document}

% TODO: Navedite broj rada.
\thesisnumber{000}

% TODO: Navedite naslov rada.
\title{Web-aplikacija za rješavanje programskih zadataka s elementima društvene mreže}

% TODO: Navedite vaše ime i prezime.
\author{Nikola Kešćec}

\maketitle

% Ispis stranice s napomenom o umetanju izvornika rada. Uklonite naredbu \izvornik ako želite izbaciti tu stranicu.
\izvornik

% Dodavanje zahvale ili prazne stranice. Ako ne želite dodati zahvalu, naredbu ostavite radi prazne stranice.
\zahvala{Srdačno se zahvaljujem svojem mentoru, prof. dr. sc Igoru Mekteroviću na cijenjenoj pomoći, savjetovanju i stručnom usmjeravanju. Također se zahvaljujem i Hermanu-Zvonimiru Došiloviću na pomoći i uputama oko Judge0-a bez kojeg ovaj završni rad ne bi bio moguć.}

\tableofcontents

\chapter{Uvod}
Sveprisutnost računalnih sustava tijekom početka drugog tisućljeća uveliko je doprinijela eksponencijalnom porastu zanimacije mlađe populacije prema svemu što je povezano s računalima, a tako i prema značajnom dijelu računarstva, činu \textbf{programiranja}. Programiranje \engl{programming} možemo opisati kao čin dizajniranja i razvoja izvršnog računalnog programa čija je svrha uspješno postizanje specifičnog računskog rezultata ili odrada specifičnog zadatka.\\
Iako su već početkom tisućljeća računalni sustavi bili poprilično sveprisutni, neusporedivo je koliko je njihova zastupljenost u svakodnevnom ljudskom životu porasla. Porastom njihove zastupljenosti porasla je potražnja za računalnom sklopovskom podrškom \engl{hardware}, a s njom i potreba za kvalitetnom programskom podrškom \engl{software}. Kvalitetna programska podrška rezultat je godina učenja, razmatranja i pisanja programskog koda, a kao i svaka druga vještina treba se održavati. Održavanje je moguće na mnogo načina, a među popularnijim načinima usavršavanja i održavanja vještosti pisanja programskog koda jest rješavanje algoritamskih zadataka te potom rangiranje tih rješenja \engl{competitive programming}.\\
Prije spomenuta zainteresirana neiskusna mladež teško da će svojevoljno uči na događanja na kojima se provodi ispitivanje vještina pisanja dobrog programskog koda jer dobar dio osoba koje sudjeluju na takvim događanjima često su već vrlo vješte. Stoga je pojava internetskih stranica koje pružaju uvid u programiranje i rješavanje programskih zadataka bila dobrodošla novost. Ipak, često riječ iskusnog programera neiskusnom programeru može biti neprocjenjiva. Iz tog razloga spoj internetskih stranica koje pružaju socijalne usluge, takozvane socijalne mreže \engl{social networks} i stranica koje pružaju mogućnost rješavanja programskih zadataka kombinacija je koja može biti značajna mladeži koju pisanje programskog koda interesira, ali i programerima koji bi htjeli kroz rješavanje zadataka održavati svoje stečeno znanje i dodatno ga usavršiti. Upravo je sinergičan produkt kombinacija osnovnih elemenata stranica socijalnih mreža i stranica s tematikom programskog koda tema ovog završnog rada.

\chapter{Postojeća programska rješenja te uvođenje socijalnih elemenata}
Već postojeća rješenja, njihove značajke te uvođenje socijalnih elemenata

\chapter{Korisnički zahtjevi}
Već postojeća rješenja, njihove značajke te uvođenje socijalnih elemenata

\chapter{Korištene tehnologije}
Već postojeća rješenja, njihove značajke te uvođenje socijalnih elemenata

\chapter{Pomoćne tehnologije}
Već postojeća rješenja, njihove značajke te uvođenje socijalnih elemenata

\chapter{Arhitektura rješenja}
Već postojeća rješenja, njihove značajke te uvođenje socijalnih elemenata

\chapter{Budući razvitak}
Već postojeća rješenja, njihove značajke te uvođenje socijalnih elemenata

\chapter{Zaključak}
Zaključak.

\bibliography{literatura}
\bibliographystyle{fer}

\begin{sazetak}
Sažetak na hrvatskom jeziku.

\kljucnerijeci{Ključne riječi, odvojene zarezima.}
\end{sazetak}

% TODO: Navedite naslov na engleskom jeziku.
\engtitle{Title}
\begin{abstract}
Abstract.

\keywords{Keywords.}
\end{abstract}

\end{document}
